\noindent\fcolorbox{black}{gray}{\makebox[\textwidth][l]{\textcolor{white}{\textsc{\textbf{Data management}}}}}

\section{Data management}\label{data}
\textit{Responsible data management is part of good research. For the collection/generation of data and the analysis of these data timely measures need to be taken to ensure its storage and later reuse. This means that prior to the start of the research project researchers must ascertain a) if the project can make use of available data from third parties, b) which project data could be relevant for reuse and c) how these data can be stored so that they are suitable for reuse.
After a proposal has been awarded funding the researcher will draw up a detailed data management plan. Please consult the explanatory notes in paragraph 6.3 of the call for proposals.}

\subsection{Will data be collected or generated that is suitable for reuse}

\begin{tabular}{|c|l|p{11cm}|}
\hline
	$\Box$ & Yes & \textit{Please answer questions \ref{data_during} to \ref{data_secure}.} \\
\hline
\hline
	$\Box$ & No & \textit{Please explain below why the research will not result in reusable data or data that cannot be stored or data that for other reasons are not relevant for reuse.} \\
\hline
\end{tabular}

\subsection{Where will the data be stored \textit{during the research}?}\label{data_during}

\subsection{After the project has been completed, how will the data be stored for the long-term and how will the data be made available for the use by third parties? For whom will the data be accessible?}

\subsection{Which facilities (ICT, (secure) archive, refrigerators, or legal expertise) do you expect will be needed for the storage of data during and after the research? Are these facilities available?}\label{data_secure}\footnote{ICT facilities for data storage are considered to be facilities such as data storage capacity, bandwidth for data transport and calculating power for data processing.}